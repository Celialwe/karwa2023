\problemname{\problemyamlname}

%\illustration{0.3}{image.jpg}{Caption of the illustration (optional). CC BY-NC 2.0 by X on Y}
% Source: URL to image.

% optionally define variables/limits for this problem
\newcommand{\maxa}{123456789}

Alexis recherche deux coéquipiers pour participer au karwa. Il dispose d'une liste de $n$ personnes sans équipe, triées par une identifiant unique.
Alexis possède un nombre porte-bonheur $k$ et il souhaite savoir s'il est possible de trouver deux coéquipiers dont la somme des identifiants est égale à $k$.

\begin{Input}
	Les entrées consiste de : 
	\begin{itemize}
		 \item Une ligne avec deux entiers : $n$ ($2 \le n \le 10^6$) représentant le nombre de personnes seules, et $k$ ($2 \le k \le 10^9$) représentant le nombre porte-bonheur d'Alexis.
		\item Une seconde ligne contient $n$ entiers $x_1, x_2, \dots, x_N$ ($1 \le x_i < k$) en ordre croissant, représentant les identifiants des personnes sans équipe.
	\end{itemize}
\end{Input}

\begin{Output}
	Si Alexis peut trouver deux coéquipiers dont la somme des identifiants est égale à $k$, le programme doit afficher ``\verb|yes|'', Sinon, il doit afficher ``\verb|no|''.
\end{Output}
