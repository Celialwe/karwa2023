\problemname{\problemyamlname}

%\illustration{0.3}{image.jpg}{Caption of the illustration (optional). CC BY-NC 2.0 by X on Y}
% Source: URL to image.

% optionally define variables/limits for this problem
\newcommand{\maxa}{123456789}

Alexis recherche deux coéquipiers pour participer au karwa. Il dispose d'une liste de N personnes sans équipe, triées par une identifiant unique. 
Alexis possède un nombre porte-bonheur K et il souhaite savoir s'il est possible de trouver deux coéquipiers dont la somme des identifiants est égale à K.


\begin{Input}
    The input consists of:
    \begin{itemize}
        \item La première ligne contient deux entiers : N (2 <= N <= $1^6$) représentant le nombre de personnes seules, et K (2 <= K <= $1^9$) représentant le nombre porte-bonheur d'Alexis.

        \item La seconde ligne contient N entiers $x_1$, $x_2$, ..., $x_N$ (1 <= $x_i$ < K) en ordre croissant, représentant les identifiants des personnes sans équipe.
        
    \end{itemize}
\end{Input}

\begin{Output}
    Si Alexis peut trouver deux coéquipiers dont la somme des identifiants est égale à K, le programme doit afficher "yes". Sinon, il doit afficher "no".
\end{Output}
