\problemname{\problemyamlname}

%\illustration{0.3}{image.jpg}{Caption of the illustration (optional). CC BY-NC 2.0 by X on Y}
% Source: URL to image.

% optionally define variables/limits for this problem

Jérôme et ses deux coéquipiers doivent se rendre aux KARWa à l'UMons. Malheureusement, ils se sont trompés d'université et ont atterri à l'UCLouvain !
Ils sont pressés par le temps et ont à leur disposition une carte avec N intersections et M liaisons bidirectionnelles, ayant chacune un poids W.
Pouvez-vous déterminer s'ils auront le temps d'arriver à l'heure ?

L'UCLouvain est représentée par l'intersection 1 et l'UMons par l'intersection N. Un chemin est garanti entre les deux universités.

\begin{Input}
    L'entrée consiste :
    \begin{itemize}
        \item Deux entiers N et M (2 <= N, M <= $10^6$), où N est le nombre d'intersections.
        \item  M lignes contenant chacune 3 entiers u, v et w, représentant une liaison bidirectionnelle entre 1 <= u <= N et 1 <= v <= N, avec un poids 1 <= w <= $10^6$.
    \end{itemize}
\end{Input}

\begin{Output}
    Une ligne contenant un entier L, la longueur du plus court chemin entre 1 et N.
\end{Output}
