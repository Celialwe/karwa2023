\begin{frame}
    \frametitle{\problemtitle}
    \begin{itemize}
        % TODO: Remove this comment when you're done writing the solutions.
        \item<+-> \textbf{Problème:} Étant donné une matrice booléenne (1 : responsable, 0 : candidats), trouver la plus grande sous-matrice ne contenant que des 1.
        \item<+-> Un algorithme naïf qui itère sur toutes les sous-matrices est en $\mathcal{O}(n^2m^2)$, trop long.
        \item<+-> On peut s'en sortir en $\mathcal{O}(nm)$ avec un seul parcours de la matrice !
        \item<+-> À chaque case $(i,j)$, on considère les tables
                  \begin{itemize}
                     \item $height[j]$ : le nombre de $1$ consécutifs sur la colonne $j$ en tenant compte des précédentes lignes $i-1,\dots$,
                     \item $left[j]$ : la colonne la plus à gauche à partir de laquelle on a uniquement des $1$, de $left[j]$ à $j$.
                     \item $right[j]$ : la colonne la plus à droite jusqu'à laquelle on a uniquement des $1$, de $j$ à $right[j]$.
                  \end{itemize}
        \item<+-> Initialement, $height[j]=0$, $left[j]=0$ et $right[j]=m$.
    \end{itemize}
\end{frame}

\begin{frame}
    \frametitle{\problemtitle}
    \begin{itemize}
        \item<+-> Pour une ligne $i$ :
                  \begin{itemize}
                     \item on parcourt chaque colonne $j$, si $M_{i,j}=1$, on actualise $height[j]\leftarrow height[j]+1$, sinon c'est $0$,
                     \item $left$ et $right$ utilisent un balayage (gauche à droite ou droite à gauche) en retenant le nombre de $1$ consécutifs qu'on a vu, puis en remettant la valeur de la table à $0$ si on croise $0$ et pas $1$,
                     \item<+-> une fois qu'on a fini la ligne $i$, on calcule le nombre de $1$ dans un sous-rectangle pour chaque colonne $j$ : \[MaxArea \leftarrow \max(MaxArea, height[j]\cdot (right[j]-left[j])).\]
                  \end{itemize}
    \end{itemize}
    % \solvestats
\end{frame}