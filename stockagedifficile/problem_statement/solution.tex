\begin{frame}
    \frametitle{\problemtitle}
    \begin{itemize}
        \item<+-> \textbf{Problème:} Trouver le nombre de positions des $n$ boissons dans une grille de $n\times 2$, avec $n \le 10^{18}$.
        \item<+-> Avec des connaissances ou des exemples, on remarque que cela revient à calculer le $n+1$\textsuperscript{ème} nombre de fibonacci $f_{n+1}$.
        \item<+-> La plupart des algorithmes ne fonctionnent pas, $10^{18}$ est une borne trop grande. Il faut impérativement utiliser une approche en $\mathcal{O}(\log n)$ !
        \item<+-> Il existe la méthode avec l'exponentiation matricielle :
        \[
        \begin{pmatrix}
        1 & 1\\
        1 & 0
        \end{pmatrix}^n =
        \begin{pmatrix}
        f_{n+1} & f_{n}\\
        f_{n} & f_{n-1}
        \end{pmatrix}
        \]
        \item<+-> Grâce à l'exponentiation optimisée (si $n$ est pair, on prend le carré de la matrice et $n\leftarrow n/2$, sinon on multiplie une fois et $n\leftarrow n-1$), on obtient un algorithme en temps logarithmique $\mathcal{O}(\log n)$.
        \item<+-> N'oubliez pas d'opérer modulo $7+10^9$ !
    \end{itemize}
    % \solvestats
\end{frame}
