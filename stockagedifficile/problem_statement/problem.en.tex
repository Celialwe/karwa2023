\problemname{\problemyamlname}

%\illustration{0.3}{image.jpg}{Caption of the illustration (optional). CC BY-NC 2.0 by X on Y}
% Source: URL to image.

% optionally define variables/limits for this problem
Pour se désaltérer après le concours, l'équipe organisatrice du KARWa a prévu un grand
nombre de boissons alcoolisées stockées dans un grand frigo de $n$ emplacements
de large et deux emplacements de haut. Vous pouvez stocker chaque boisson de
deux façons : soit verticalement, et elle prend ainsi un emplacement de long sur
deux emplacements de haut, soit horizontalement, et elle prend ainsi deux
emplacements de long sur un emplacement de haut. De combien de manières
différentes est-il possible de remplir le frigo ?

Attention, la réponse peut être très grande. Nous vous demandons donc d'output
la réponse modulo $7+10^9$.

\begin{Input}
	Un entier $n$ ($1 \le n \le 10^{18}$), le nombre d'emplacements dans le frigo sur sa longueur.
\end{Input}

\begin{Output}
	Le nombre de manières de stocker des boissons dans le frigo modulo $7+10^9$.
\end{Output}
