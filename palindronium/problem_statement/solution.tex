\begin{frame}
    \frametitle{\problemtitle}
    \begin{itemize}
        \item<+-> \textbf{Problème:} Optimiser l'aire d'un triangle rectangle $(a,b,c)$ en conservant le même périmètre
        \item<+-> Pour l'hypoténuse $c$ fixée, l'aire est maximale si $a=b$ (simple problème d'optimisation par dérivation).
        \begin{itemize}
            \item $a^2 + b^2 = h^2$
            \item $0.5 \cdot a\cdot b = A$
            \item $b = \sqrt[2]{h^2 - a^2}$ 
            \item $\frac{\partial A}{\partial a} = \frac{h^2-2a^2}{2\sqrt[2]{h^2-a^2}}$
            \item $a=b=\frac{h}{\sqrt[2]{2}}$
        \end{itemize}
        \item<+-> Si on ignore la contrainte d'entier palindromique sur $c$, le plus grand rectangle possible est $(n,n,n\sqrt{2})$ qui prend la moitié du carré $n\times n$.
        \item<+-> Avec la contrainte, on itière avec $c$ de $\lfloor n\sqrt{2} \rfloor$ à $0$ : prendre le premier palindrome rencontré et afficher l'aire.
        \item<+-> Pour l'hypoténuse connue $c$, un autre côté est de longueur $c/\sqrt{2}$ donc l'aire est donnée par \[\frac{1}{2}\cdot\left(\frac{c}{\sqrt{2}}\right)^2 = \frac{c^2}{4}.\]
%Attention à bien stopper la boucle dès qu'on trouve un palindrome. Aussi, on doit bien arrondir à l'inférieure et pas round ni ceil(n*sqrt(2)), ça casse si c'est un palindrome.

    \end{itemize}
    % \solvestats
\end{frame}
