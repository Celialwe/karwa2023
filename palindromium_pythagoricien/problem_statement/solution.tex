\begin{frame}
    \frametitle{\problemtitle}
    \begin{itemize}
        % TODO: Remove this comment when you're done writing the solutions.
        \item<+-> \textbf{Problem:} Optimiser l'aire d'un triangle rectangle $(a,b,c)$ en conservant le même périmètre
        \item<+-> $\Rightarrowa a=b$, triangle isocèle.
        \item<+-> Si on ignore la contrainte d'entier palindromique sur $c$, le plus grand rectangle possible est $(n,n,n*\sqrt{2})$.
        \item<+-> Avec la contrainte, $c$ va de $floor(n*sqrt(2))$ à $0$ : prendre le premier palindrome rencontré et afficher l'aire.
        \item<+-> Pour l'hypothénuse connue $c$, un autre côté est de longueur $c/sqrt(2)$ donc l'aire est donnée par $\frac{1}{2}\cdot(\frac{c}{\sqrt{2}})^2 = \frac{c^2}{4}$.
%Attention à bien stopper la boucle dès qu'on trouve un palindrome. Aussi, on doit bien arrondir à l'inférieure et pas round ni ceil(n*sqrt(2)), ça casse si c'est un palindrome.

    \end{itemize}
    % \solvestats
\end{frame}
